 \documentclass[10pt,DIV12,a4paper,halfparskip]{scrartcl}
\usepackage{amsmath}
\usepackage[T1]{fontenc}

\newcommand{\changefont}[3]{\fontfamily{#1} \fontseries{#2} \fontshape{#3} \selectfont}

%% commands to facilitate units and temperature
\newcommand{\unit}[1]{\ensuremath{\,\mathrm{#1}}}
\newcommand{\s}[1]{\ensuremath{\,\mathrm{#1}}}
\newcommand{\cels}[1]{\ensuremath{#1^{\circ}\,\mathrm{C}}}


% ---------------------------------------------------------------------------------------
% definition of header and footer
% ---------------------------------------------------------------------------------------

\usepackage[automark,headsepline,footsepline]{scrpage2}
\clearscrheadings	% definition headings loeschen
\ihead{Thermodynamics}
\chead{McCarthy Summer School}
\ohead{2010}
\cfoot{\pagemark}
\setkomafont{pagehead}{\normalfont}	% font kopfzeile
\setkomafont{pagenumber}{\normalfont\rmfamily}	% font kopfzeile zahlen




% ---------------------------------------------------------------------------------------
% Koma-Script - Settings
% ---------------------------------------------------------------------------------------
\addtokomafont{caption}{\small}
\setkomafont{captionlabel}{\sffamily\bfseries}
% \setkomafont{caption}{\sffamily}
\setcapindent{1em}

\pagestyle{scrheadings}


\begin{document}

\vspace{-5em}

\title{Thermodynamics of Glaciers \\
\rule[1em]{\textwidth}{2pt}
\LARGE{\sf Exercise}
}
\date{}

\vspace{-5em}

\maketitle


\vspace{-5em}

\section{Climate history}

Air temperatures in Alaska was oscillating with a period of about 50 years and an amplitude of about 2$\,^\circ\text{C}$ between 1950 and 2000. How deep down would you be able to detect such temperature variation in stagnant ice if the accuracy of your temperature sensors are 0.01$\,\text{K}$ assuming an ice temperature of $-\cels{3}$?


\section{Melting temperature depression}

What is the pressure melting temperature at the base of Gornergletscher (Fig.9)? What does the Clausius-Clapeyron relation indicate in terms of air-saturation of the melt water? The pressure $p$ is the sum of the hydrostatic pressure and the atmosperic pressure, $p = \rho g H + p_{\text{atm}}$. Assume $p_{\text{atm}} = 75'000\,\text{Pa}$.

\section{Lake Vostok}

\begin{enumerate}
\item Describe 2 different ways how heat can be moved through a polar ice sheet.
\item What is the P\'eclet Number, and how is it useful?
\item The coldest temperature ever recorded is -\cels{89} at Vostok in East Antarctica (in July 1983). The mean annual temperature is -\cels{55}. However, deep under the ice is lake Vostok, a lake of the size of lake Ontario. Calculate the minimum geothermal flux needed for a lake to form. Possibly relevant quantities:
  \begin{itemize}
  \item Surface elevation $3488\,\text{m}$
  \item Ice thickness $3300\,\text{m}$
  \item Snow accumulation rate $2\,\text{cm}\,\text{a}^{-1}$ (water equivalent)
  \item A reasonable average thermal conductivity for the cold temperatures of the East Antarctic Ice Sheet is $k = 2.5\,\text{W}\,\text{m}^{-1}\,\text{K}^{-1}$.
  \end{itemize}
\end{enumerate}

\end{document}