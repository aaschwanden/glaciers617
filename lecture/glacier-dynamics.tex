
\documentclass[notes=hide]{beamer}


\mode<presentation>
{
  \usetheme[headline,footline]{ARSC}
  \setbeamercovered{transparent}
}

% load packages
\usepackage[english]{babel}
\usepackage[latin1]{inputenc}
\usepackage{multimedia}
\usepackage{lmodern}
\usepackage[T1]{fontenc}
%\usepackage{hyperref}
\usepackage{booktabs,pgf,verbatim}
\usepackage{springervector}
\bibliographystyle{chicago}


% NOTE:
% Instead of pgf, you can also use the familiar \includegraphics commad





% title page
\title[Glacier Dynamics] % (optional, use only with long paper titles)
{Glacier Mechanics and Thermodynamics}



\author[Aschwanden] % (optional, use only with lots of authors)
{Andy Aschwanden}
% - Give the names in the same order as the appear in the paper.
% - Use the \inst{?} command only if the authors have different
%   affiliation.

\institute[ARSC] % (optional, but mostly needed)
{
  %
  Arctic Region Supercomputing Center\\
  University of Alaska Fairbanks, USA
}
% - Use the \inst command only if there are several affiliations.
% - Keep it simple, no one is interested in your street address.




\subject{Development}

% define what is shown at the beginning of each section
\AtBeginSection[]
{
 \begin{frame}<beamer>
   \frametitle{Outline}
   \tableofcontents[currentsection,subsectionstyle=hide/show/hide]
 \end{frame}
}

% define what is shown at the beginning of each subsection
\AtBeginSubsection[]
{
 \begin{frame}<beamer>
  \frametitle{Outline}
   \tableofcontents[currentsection,currentsubsection]
 \end{frame}
}

\begin{document}

% insert titlepage
\begin{frame}
  \titlepage
  \note[item]{Good afternoon everybody and welcome to my presentation "NumMod of GlacFlow".}
  \note[item]{My name is Andy Aschwanden and I'm a PhD student of H. Blatter at IAC, working on glacier mechanics and dynamics}
  \note[item]{in the next 15 minutes, I'll present how I'm using CMP to model glacier flow and how it has change my scientific life}
\end{frame}

% insert TOC
\begin{frame}
 \frametitle{Outline}
 \tableofcontents[subsectionstyle=hide]
  %You might wish to add the option [pausesections]
  \note[item]{I've divided my talk into 4 sections...}
\end{frame}



\section{Introduction}



\begin{frame}
  \frametitle{What is a glacier?}
  \begin{itemize}[<+- | alert@+>] % some control parameters
    \item for artists, tourists: beautiful landscape
    \item Geographers: element of landscape
    \item Geologists: soft rock, sediment
    \item Hydrologists: water reservoir
    \item Climatologists: subsystem of climate system, climate archive
    \item Physicists: thermomechanical non-Newtonian fluid
    \item Mathematicians: free boundary problem in fluid dynamics
    \item Electric engineers: one sided accessible dielectricum
    \item Glaciologists: glacier
  \end{itemize} 
\end{frame}

\subsection{Basics}

\begin{frame}
  \frametitle{Glaciers for glaciologists}
  \vspace{-.1cm}
  Glacier ice: incompressible, heat-conducting non-Newtonian fluid
  \vspace{.1cm}
  \small{
  \begin{columns}
  \column[T]{6cm}
  Mass conservation
  \begin{displaymath}
  \nabla \cdot \vec v =  0
  \end{displaymath}
  Stokes equation
  \begin{displaymath}
  \nabla \cdot \left( -p\vec{\mathrm I} + 2\eta \vec {\mathrm D}\right) = \rho \vec g
  \end{displaymath}
  Temperature equation
  \begin{displaymath}
  \nabla \cdot \left(-k\nabla T\right) + \rho c_p \vec v \cdot \nabla T = 0
  \end{displaymath}
    Viscosity
  \begin{displaymath}
  \eta = \eta\left(T,\omega,\dot \varepsilon_{eff},\ldots \right)
  \end{displaymath}
  \column[T]{5cm}
    \small
    \begin{table}
    \begin{tabular}{ll}
      $\vec v$ & velocity \\
      $p$ & pressure \\
      $\rho$&  density of ice \\
      $\vec g$ & acceleration due to gravity \\
      $\vec {\mathrm I}$ & identity tensor \\
      $T$ & temperature \\
      $k$ & diffusivity of ice \\
      $c_p$ & heat capacity of ice \\
      $\eta$ & viscosity of ice \\
      $\omega$ & water content \\
      $\vec {\mathrm D}$ & $=\frac{1}{2}\left( \nabla \vec v + \nabla \vec v^{T}\right)$\\
      $\dot \varepsilon_{eff}$ & $=\left(\frac{1}{2}\mathrm{tr} \left(\vec {\mathrm D^T}\vec {\mathrm D}\right)\right)^{1/2}$\\
      & effective strain rate
    \end{tabular}
  \end{table}
  \end{columns}
  }
  \note[item]{obeys mass conservation and Stokes flow}
  \note[item]{not going into detail}
  \note[item]{viscosity depends on \ldots}
  \note[item]{non-linear relationship}  
\end{frame}



% \begin{frame}
% \frametitle{Glaciers are multiphysics pure}
%   \begin{figure}
%     {\pgfuseimage<1>{mp_energy}}%
%     {\pgfuseimage<2>{mp_hydro}}%
%     {\pgfuseimage<3>{mp}}%
%   \end{figure}
% \end{frame}



\end{document}


